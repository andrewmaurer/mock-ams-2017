\documentclass{beamer}

\usepackage{my_macros}
\usepackage{amssymb,amsthm,amsmath}
\usepackage{enumerate}
\usepackage{tikz-cd}

\theoremstyle{definition}
\newtheorem{defn}{Definition}
%\newtheorem{example}{Example}

\newcommand{\GL}{\operatorname{GL}}
\newcommand{\FF}{\mathbb{F}}
\newcommand{\Ext}{\operatorname{Ext}}
\newcommand{\ZZ}{\mathbb{Z}}
\newcommand{\spec}{\operatorname{Spec}}


\begin{document}

\title{Representation Theory via Geometry}
\author{Andrew Maurer}
\date{Summer 2017}

\frame{\titlepage}

\begin{frame}{Overview}
  \pause
  \begin{enumerate}
  \item Algebra \pause
    \begin{itemize}
    \item $G$-modules \pause
    \item Cohomology groups of a $G$-module \pause
    \item Ring structure\pause
    \item $G$-modules $\rightsquigarrow$ $H^\bullet(G;k)$-modules \pause
    \end{itemize}
  \item Geometry
    \begin{itemize}
    \item $G$ defines a variety, and $G$-modules define subvarieties \pause
    \item Operations on $G$-modules vs. operations on varieties \pause
    \item Realizability 
    \end{itemize}
  \end{enumerate}
\end{frame}

\begin{frame}{Conventions}
  \pause
  \begin{itemize}
  \item All groups will be finite, usually denoted $G$.\pause
  \item $k$ will denote a field which is:\pause
    \begin{itemize}
    \item of characteristic $p \mid |G|$\pause
    \item algebraically closed\pause
    \end{itemize}
  \item $M$ will be a $G$-module.\pause
  \item $\spec(R)$ refers to the maximal ideal spectrum of the ring $R$.\pause
  \end{itemize}
\end{frame}

\section{Algebra}

\begin{frame}{Finite groups}\pause
  Recall that a finite group is a finite set $G$ with a multiplication rule satisfying some axioms that make it behave like symmetry.\pause
  \begin{itemize}
  \item $\GL(V)$ = all linear isomorphisms of a vector space $V/\FF_q$.\pause
  \item $S_n$ = all bijections $\{1,2,\ldots,n\} \to \{1,2,\ldots,n\}$.\pause
  \item $(\ZZ/p)^r$ = elementary Abelian group of rank $r$.\pause
  \item Others
  \end{itemize}
  
\end{frame}

\begin{frame}{Modules}\pause
  Groups are modelled on symmetries. A \textbf{$G$-module} is just a vector space where $G$ acts as symmetries, i.e., linear automorphisms.\pause

  \vspace{0.2in}
  Namely, there's a map \[
    \varphi : G \to \GL(V), g \mapsto \varphi_g
  \]
\pause
  Examples:\pause
  \begin{itemize}
  \item $V$ is a $\GL(V)$ module.\pause
  \item The vector space with basis $\{e_1,\ldots,e_n\}$ is a $S_n$-module by $\varphi_\sigma(e_i) = e_{\sigma(i)}$, and extend by linearity.\pause
  \item $k$ is a $G$-module via the trivial action $\varphi_g(v) = v$.\pause
  \end{itemize}
  {\color{red} $G$-modules as defined above are really just modules for a certain ring denoted $kG$.}
\end{frame}

\begin{frame}{Cohomology groups are $\Ext$ groups}\pause
  The cohomology group $H^n(G;k)$ can be defined in many ways.\pause

  At the end of the day, the most concrete way to realize $H^n(G;M)$ is as $\Ext^n_{kG}(k,M)$.
  \[
\Ext_{kG}^n(k,M) = \big\{ 0 \to M \to E_1 \to \ldots \to E_n \to k \to 0 \big\} / \sim
  \]
  e.g., as all $n$-fold extensions of $k$ by $M$, modulo a certain equivalence relation.\pause

  Addition in the cohomology group corresponds to Baer sum of extensions.

\end{frame}

\begin{frame}{Yoneda splicing}\pause

  Realizing cohomology groups $H^n(G;k)$ in this way means we can multiply two sequences together.
  \[
    \Ext^n_{kG}(k,k) \times \Ext^m_{kG}(k,k) \to \Ext^{m+n}_{kG}(k,k)
  \]\pause
  \[
    \begin{tikzcd}[ampersand replacement=\&]
      \ldots \arrow[r] \& E_n \arrow[r] \arrow[drr,dashed] \& k \arrow[d,equals]\arrow[r] \& 0  \\
      \& 0 \arrow[r] \& k \arrow[r] \&  F_1 \arrow[r] \& \ldots 
    \end{tikzcd}
  \]\pause
  And the new sequence has the form
  \[
    0 \to k \to \underbrace{E_1 \to \ldots \to E_n \to F_1 \to \ldots \to F_m}_{m+n} \to k \to 0
  \]
  
\end{frame}

\begin{frame}{$H^\bullet(G;k) = \Ext^\bullet_{kG}(k,k)$ is a ring}\pause
  Under this $\Ext^n_{kG}(k,k) \times \Ext^m_{kG}(k,k) \to \Ext^{n+m}_{kG}(k,k)$ association, \[
    \Ext^\bullet_{kG}(k,k) = \bigoplus \Ext^n_{kG}(k,k)
  \]
  becomes a graded ring. \pause What can we say about it? \pause
  \begin{itemize}
  \item It is \emph{graded-commutative}, i.e., $\alpha \cdot \beta = (-1)^{\bar \alpha \cdot \bar \beta} \beta \cdot \alpha$.\pause
  \item (Evens 1961) It is finitely-generated over $k$.\pause
  \end{itemize}
  Define a commutative, finitely generated ring:
  \[
    H^c(G;k) = \begin{cases}
      H^\bullet(G;k) &\text{ if } p = 2 \\
      \bigoplus H^{2n}(G;k) &\text{ if } p > 2
      \end{cases}
   \] 
  
 \end{frame}

  \begin{frame}{One more thing about rings}\pause
   $\Ext^\bullet_{kG}(M,M)$ is a ring in the exact same way.\pause

   \vspace{0.2in}
   $\Ext^\bullet_{kG}(M,M)$ is a graded module for $H^\bullet(G;k)$ (or $H^c(G;k)$) using the tensor product.\pause

   \vspace{0.2in}
   So $\Ext^\bullet_{kG}(M,M)$ has an annihilator $I_M \trianglelefteq H^\bullet(G;k)$ which is a homogeneous ideal.
 \end{frame}

 \section{Geometry}

 \begin{frame}{Algebra $\rightsquigarrow$ Geometry}\pause
   We have a commutative, finitely generated $k$-algebra $H^c(G;k)$ which captures the representation theory of $G$ in some way, and a way to associate ideals to $G$-modules.\pause

   \vspace{0.2in}
   What information is present in the affine variety $V_G(k) = H^\bullet(G;k)$ and its conical subvarieties $V_G(M) = Z(I_M) \subseteq V_G(k)$?\pause

   \vspace{0.2in}
   \begin{theorem}
     $V_G(M) = \{0\}$ if and only if $M$ is projective. \pause
   \end{theorem}
 \end{frame}

 \begin{frame}{Operations on modules and varieties} \pause
   \begin{theorem}
     \begin{enumerate}
     \item $V_G(M_1 \oplus M_2) = V_G(M_1) \cup V_G(M_2)$ \pause
     \item $V_G(M_1 \otimes M_2) = V_G(M_1) \cap V_G(M_2)$ \pause
     \item $V_G(M^*) = V_G(M)$ \pause
     \end{enumerate}
   \end{theorem}

   How are these proved? \pause Main techinique is by writing
   \[
V_G(M) = \bigcup_{\substack{E \leq G \\ \text{elem}}} \varphi_E ( V_E(M))
\]
$\varphi_E$ is a natural map obtained by thinking of $M$ as an $E$-module.
\pause
\vspace{0.2in}
Modules $L_\zeta$ so that $V_G(L_\zeta) = Z(\zeta)$, where $\zeta \in H^\bullet(G;k)$.
\end{frame}

\begin{frame}{Realizability}
\pause
  \begin{theorem}
    Every conical subvariety arises as $V_G(M)$ for some module $M$.
  \end{theorem}
\pause
  \vspace{0.4in}
  \begin{theorem}
    If $V_G(M) = V_1 \cup V_2$ with $V_1 \cap V_2 = \{0\}$, then there exist modules $M_1$ and $M_2$ with $V_G(M_1) = V_1$ and $V_G(M_2) = V_2$.
  \end{theorem}
  
\end{frame}
\end{document}
